% Chapter Template

\chapter{Related work} % Main chapter title


\label{Chapter3} % Change X to a consecutive number; for referencing this chapter elsewhere, use \ref{ChapterX}

%----------------------------------------------------------------------------------------
%	SECTION 1
%----------------------------------------------------------------------------------------

\section{Automated Test Input Generation for Android:
Are We There Yet}

//TODO esta sección pueden tratarse de este paper, es bastante similar, diría que hicieron más o menos lo mismo que yo, lo diferente es el enfoque que ellos le están dando al estudio, están comparando esas herramientas pero no dan el marco para seguir comparando las siguientes, no dan las herramientas, por el contrario están usando otras herramientas que requieren más configuración. 
Por qué mi idea o mi estudio es mejor? porque proporciono herramientas para que los desarrolladores puedan volver a ejecutar esas comparaciones de forma fácil y rápida, permitiendoles decidir la mejor herramienta para sus projectos.

Las herramientas que usé son diferentes a escepción de Monkey. Es monkey diferente en los dos estudios? Por qué?

\section{Droidbot}
Droidbot itself puede ser un related work porque tuvieron que haber hecho comparacones contra otras aplicaciones para decir que esta es mejor en algún sentido. Lo bueno de esas comparaciones es que me permiten decir que estos estudios no son imparciales y que pueden estar sesgados por alguna razón desconocida y hasta inintencionada por parte de los autores. Que aunque en este estudio se utilizaron solo aplicaciones que servian con RIP, con los resultados se hace evidente que no se pretendía en ningún momento favorecer los resultados de esa herramienta.

\section{RIP}
Aunque no se ha publicado un paper sobre esto (no sé si lo pueda poner) diciendo que representa que aún con la cantidad de herramientas existentes los desarrolladores e investigadores siguen haciendo nuevas herramientas en busca de nuevas formas de exploración, más eficientes o inclusive mezclando las ya existentes con el fin de lograr la mejor exploración, así como la mayor cantidad de datos importantes para los desarrolladores. Con estos resultados las personas pueden saber que lo que están haciendo está bien o esta mal y sabiendo el camino que deben seguir de ahora en adelante.

\section{This approach}

Si se puede escribir así porque no es related work en nada en realidad. tal vez seguir la sección? pero si sigo la sección es como si fuera de esa sección adentro. más bien podría como enumerar y al final concluir lo de esta sección. 
Me parece mejor.

