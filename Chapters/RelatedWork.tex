% Chapter Template

\chapter{Related work} % Main chapter title


\label{Chapter3} % Change X to a consecutive number; for referencing this chapter elsewhere, use \ref{ChapterX}

%----------------------------------------------------------------------------------------
%	SECTION 1
%----------------------------------------------------------------------------------------

Every exploration tool provides different numbers, and comparisons made during their creation, this in order to show their advantages, but not for sure their weaknesses. This information does not allow developers neither researchers to know what are the best tools as of now or how good a tool matches their projects. 
%TODO Cite them propertly here.

Shauvik Roy Choudhary, et al., \MARIO{[CITE]} give about the strengths and weaknesses of seven tools in their study "Automated Test Input Generation for Android:
Are We There Yet?". They evaluate the tools using four metrics i. ease
of use, ii. ability to work on multiple platforms, iii. code coverage, and iv.
ability to detect faults. 14 Tools and 68 applications were used in total in their study. Running 10 times every application in seven of the 14 tools for a maximum of 60 minutes.

Moreover, different tools have been developed since Shauvik Roy Choudhary, et al. \MARIO{[CITE]} study was made. One example is Firebase Test Lab, which started to allowing the automatic testing in the cloud. This type of application was not taken into account.

\MARIO{Please cite and discuss the papers with the rippers (droidbot, crawldroid, etc):
\\
https://ieeexplore.ieee.org/abstract/document/8530063/
\\
 https://ieeexplore.ieee.org/abstract/document/8094439
\\	
https://ieeexplore.ieee.org/abstract/document/8094467/
\\
https://ieeexplore.ieee.org/document/7102609
\\
https://www.sciencedirect.com/science/article/pii/S0164121216300140
\\
https://link.springer.com/article/10.1007/s11219-018-9418-6?shared-article-renderer
}
