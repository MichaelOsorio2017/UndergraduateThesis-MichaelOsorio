% Chapter Template

\chapter{Introduction} % Main chapter title

\label{Chapter1} % Change X to a consecutive number; for referencing this chapter elsewhere, use \ref{ChapterX}
 
Parafrasear lo que dice aquí, decir que este gran número de aplicaciones por fortuna a llevado a un gran número de estudios para mejorar la cobertura de código que se logra durante las pruebas automáticas, y también cubir diferentes estados del celular como modo avión con carga sin carga y demás, dada la gran cantidad de herramientas, los desarrolladores pueden sentirse sobrecargados, pueden exitir muchas opciones y pueden que no se elija la mejor herramienta. 
Los desarrolladores necesitan formas de elegir la mejor herramienta que se adapte lo mejor posible a sus necesidades.
 
Mobile markets have pushed and promoted the raising of an interesting phenomenon that has permeated not only developers culture, but also human beings’ daily life activities. Mobile devices, apps, and services are helping companies and organizations to make “digital transformation” possible through services and capabilities that are offered ubiquitously and closer to the users. Nowadays, mobile apps and devices are the most common way for accessing those services and capabilities; in addition, apps and devices are indispensable tools for allowing humans to have in their phones, computational capabilities that make life better and easier.

The mobile apps phenomenon has also changed drastically the way how practitioners design, code, and test apps.  Mobile developers and testers face critical challenges on their daily life activities such as (i) continuous pressure from the market for frequent releases of high quality apps, (ii) platform fragmentation at device and OS levels, (iii) rapid platform/library evolution and API instability, and (iv) an evolving market with millions of apps available for being downloaded by ends users \cite{joorabchi2013real,palomba2018crowdsourcing}. Tight release schedules, limited developer and hardware resources, and cross-platform delivery of apps, are common scenarios when developing mobile apps \cite{joorabchi2013real}. Therefore, reducing the time and effort devoted to software engineering tasks while producing high quality mobile software is a ``precious’’ goal.

Both practitioners and researchers, have contributed to achieve that goal, by proposing approaches, mechanisms, best practices, and tools that make the development process more agile. For instance, cross-platform languages and frameworks (e.g., Flutter, Ionic, Xamarin, React Native) contribute to reducing the development time by providing developers with a mechanism for building Android and iOS versions of apps in a write-one-run-anywhere way \cite{joorabchi2013real,fazzini2017automated}. Automated testing approaches help testers to increase the apps' quality and reduce the detection/reporting time \cite{choudhary2015automated,kochhar2015understanding,linares2017continuous}. 
Automated categorization of reviews also helps developers to select relevant information, issues, features and sentiments, from large volume of review that are posted by users \cite{palomba2018crowdsourcing,villarroel2016release,di2016would}. Moreover, approaches for static analysis, are helping developers to early detect different types of bugs and issues that without the automated support could be time consuming for developers --- when doing the analysis manually \cite{li:IST2017}. 
Both static and dynamic analyses have been used with the aforementioned approaches, with a special preference for static analysis on source code.  

The developers community is quickly moving towards using cloud-services and crowd-sourced services for software engineering tasks \cite{Leicht2017IEEESoftware, stol2017crowdsourcing}; using those services is becoming a common practice of mobile developers who want to reduce costs and the time devoted for an activity. For example, the Firebase Test Lab platform \cite{firebase} provides automated testing services, in particular, it automatically executes/explores a given app (provided by the developer as an Android APK file), and reports crashes found on a devices matrix that is selected by the user. However, the lack of knowledge of source code internals imposes a limitation on the usefulness and completeness of the results reported back to the users.

%----------------------------------------------------------------------------------------
%	SECTION 1
%----------------------------------------------------------------------------------------

\section{Problem Statement}

// TODO explicar el problema que se quiere solucionar. Para mi es el comparar las diferentes herramientas puede tomar tiempo y elegir la correcta para un proyecto o para una aplicación puede tomar tiempo valioso


\section{Thesis  Goals}

//TODO comparar herramientas de exploración automática, dar un punto de vista objetivo ofreciendo datos para que los desarrolladores puedan elegir la mejor herramienta para sus proyectos.

Medir cobertura de código



\section{Thesis contribution}

//TODO

\section{Document Structure}

// TODO Hacer al final porque no se sabe la estructura antes de

