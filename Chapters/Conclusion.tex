% Chapter Template

\chapter{Conclusion} % Main chapter title
\label{Chapter5}
% Change X to a consecutive number; for referencing this chapter elsewhere, use \ref{ChapterX}

With the results showed in sections \ref{sec:coverageResults} and \ref{sec:errorResults} there are enough data to conclude that: The best tool for reach high method coverage is Firebase Test Lab, which also is capable of find some errors while exploring.

Nevertheless, in a testing environment, the high method coverage is not that important if no errors are found, for that reason, Monkey is a better option. Even when this tool has no complex architecture, and is the default tool provided with Android SDK Tools, in this study is shown that its relation between method coverage reached and number of errors found is better than the other tools presented in this document. 

Another conclusion for this study is that, a more complex exploration strategy not always leads to better coverage and higher numbers of errors discovered.

This results allow developers and researcher to validate their decisions when selecting an automatic exploration tool for Android applications, as well as give them an idea of what is still missing when regarding to automatic exploration of Android applications.