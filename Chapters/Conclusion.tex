% Chapter Template

\chapter{Conclusions and Future Work} % Main chapter title
\label{Chapter5}
% Change X to a consecutive number; for referencing this chapter elsewhere, use \ref{ChapterX}

With the results showed in sections \ref{sec:coverageResults} and \ref{sec:errorResults} there are enough data to conclude that: The best tool for reach high method coverage is Firebase Test Lab, which also is capable of find some errors while exploring.

Nevertheless, in a testing environment, the high method coverage is not that important if no errors are found, for that reason, Monkey is a better option. Even when this tool has no complex architecture, and is the default tool provided with Android SDK Tools, in this study is shown that its relation between method coverage reached and number of errors found is better than the other tools presented in this document. 

Another conclusion for this study is that, a more complex exploration strategy not always leads to better coverage and higher numbers of errors discovered.

This results allow developers and researcher to validate their decisions when selecting an automatic exploration tool for Android applications, as well as give them an idea of what is still missing when regarding to automatic exploration of Android applications.

As more and more exploration tools are designed and implemented, as well as exploration strategies there is the need of repeat this research periodically owing to provide information of the newest tool to developers and researchers. This study also will allow researchers to know what are the next steps for reaching better testing tools regarding the automatic exploration. 

Besides, the work flow designed for this study and detailed in section \ref{sec:generalApproach} can be refined, extended and enhance so as to achieve the standardization of the validation of new exploration tools for Android apps. 

Furthermore, because of time limitations, and resources, the number of applications used in this study was not as high as was expected at the beginning. So, in order to extend this study, more applications and different tools can be use, as well as use human exploration to compare the results of a human-being against the automatic exploration tools can be made.

Finally, the tool designed for this study InstruAPK (\ref{sec:instruAPK}) and CoverageAnalyzer (\ref{sec:ca}) can have several improvements. InstruAPK can only instruments methods that are being called inside the source code, avoiding the instrumentation of dead code, which for now could be rising the number of instrumented methods and limiting the accuracy of the coverage reports.