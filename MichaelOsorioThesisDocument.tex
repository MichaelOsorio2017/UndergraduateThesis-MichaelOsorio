%----------------------------------------------------------------------------------------
%	PACKAGES AND OTHER DOCUMENT CONFIGURATIONS
%----------------------------------------------------------------------------------------

\documentclass[
12pt, % The default document font size, options: 10pt, 11pt, 12pt
%oneside, % Two side (alternating margins) for binding by default, uncomment to switch to one side
english, % ngerman for German
onehalfspacing, % Single line spacing, alternatives: singlespacing, onehalfspacing or doublespacing
%draft, % Uncomment to enable draft mode (no pictures, no links, overfull hboxes indicated)
nolistspacing, % If the document is onehalfspacing or doublespacing, uncomment this to set spacing in lists to single
liststotoc, % Uncomment to add the list of figures/tables/etc to the table of contents
%toctotoc, % Uncomment to add the main table of contents to the table of contents
parskip, % Uncomment to add space between paragraphs
%nohyperref, % Uncomment to not load the hyperref package
headsepline, % Uncomment to get a line under the header
%chapterinoneline, % Uncomment to place the chapter title next to the number on one line
consistentlayout, % Uncomment to change the layout of the declaration, abstract and acknowledgements pages to match the default layout
]{MastersDoctoralThesis} % The class file specifying the document structure


\usepackage[utf8]{inputenc} % Required for inputting international characters
\usepackage[T1]{fontenc}

\usepackage{mathpazo} % Use the Palatino font by default
\usepackage{color}
\usepackage{subfigure}
\usepackage{array}
\usepackage{listings}
\usepackage{amssymb}
\usepackage{ifthen}
\usepackage{float}
\usepackage{lscape}
\newcolumntype{C}[1]{>{\centering\let\newline\\\arraybackslash\hspace{0pt}}m{#1}}
\newcolumntype{L}[1]{>{\centering\let\newline\\\arraybackslash\hspace{0pt}}p{#1}}
\usepackage{hyperref}
\lstset{numbers=left, basicstyle=\footnotesize, frame=single, breaklines=true, breakatwhitespace=true}

\usepackage[backend=bibtex]{biblatex} % Use the bibtex backend with the authoryear citation style (which resembles APA)

\addbibresource{example.bib}

\usepackage{amsmath}

\usepackage[autostyle=true]{csquotes} % Required to generate language-dependent quotes in the bibliography

\newcommand{\projectName}{\textbf{ProjectName} }

%----------------------------------------------------------------------------------------
%	MARGIN SETTINGS
%----------------------------------------------------------------------------------------

\geometry{
	paper=a4paper, % Change to letterpaper for US letter
	inner=2.5cm, % Inner margin
	outer=3cm, % Outer margin
	bindingoffset=.5cm, % Binding offset
	top=1.5cm, % Top margin
	bottom=1.5cm, % Bottom margin
	%showframe, % Uncomment to show how the type block is set on the page
}

%----------------------------------------------------------------------------------------
%	THESIS INFORMATION
%----------------------------------------------------------------------------------------

\thesistitle{Comparison and Analysis Between Automatic Exploration Tools for Android Applications} % Your thesis title, this is used in the title and abstract, print it elsewhere with \ttitle
\supervisor{Mario \textsc{Linares-Vásquez}} % Your supervisor's name, this is used in the title page, print it elsewhere with \supname
\examiner{} % Your examiner's name, this is not currently used anywhere in the template, print it elsewhere with \examname
\degree{Bachelor in Software and Computer Engineering} % Your degree name, this is used in the title page and abstract, print it elsewhere with \degreename
\author{Michael \textsc{Osorio-Riaño}} % Your name, this is used in the title page and abstract, print it elsewhere with \authorname
\addresses{} % Your address, this is not currently used anywhere in the template, print it elsewhere with \addressname

\subject{Biological Sciences} % Your subject area, this is not currently used anywhere in the template, print it elsewhere with \subjectname
\keywords{} % Keywords for your thesis, this is not currently used anywhere in the template, print it elsewhere with \keywordnames
\university{\href{http://uniandes.edu.co}{University of Los Andes}} % Your university's name and URL, this is used in the title page and abstract, print it elsewhere with \univname
\department{\href{http://sistemas.uniandes.edu.co}{Systems and Computing Engineering Department}} % Your department's name and URL, this is used in the title page and abstract, print it elsewhere with \deptname
\group{\href{https://ticsw.uniandes.edu.co/}{TICSw research group}} % Your research group's name and URL, this is used in the title page, print it elsewhere with \groupname
\faculty{Faculty of Engineering} % Your faculty's name and URL, this is used in the title page and abstract, print it elsewhere with \facname

\AtBeginDocument{
\hypersetup{pdftitle=\ttitle} % Set the PDF's title to your title
\hypersetup{pdfauthor=\authorname} % Set the PDF's author to your name
\hypersetup{pdfkeywords=\keywordnames} % Set the PDF's keywords to your keywords
}
\input{macro}

\begin{document}

\frontmatter % Use roman page numbering style (i, ii, iii, iv...) for the pre-content pages

\pagestyle{plain} % Default to the plain heading style until the thesis style is called for the body content

%----------------------------------------------------------------------------------------
%	TITLE PAGE
%----------------------------------------------------------------------------------------

\begin{titlepage}
\begin{center}


% {\scshape\LARGE \univname\par}\vspace{1.5cm} % University name
\begin{figure}[h]
\centering
\includegraphics[width=0.4\textwidth]{../Figures/logoUniandes}
\end{figure}
%\vspace*{.07\textheight}
%\textsc{\Large Master Thesis}\\
\vspace{0.3cm} % Thesis type

\HRule \\[0.4cm] % Horizontal line
{\huge  \ttitle\par}\vspace{0.4cm} % Thesis title
\HRule \\[1cm] % Horizontal line
 
\begin{minipage}[t]{0.4\textwidth}
\begin{flushleft} \large
\emph{Author:}\\
\href{https://michaelosorio2017.github.io/}{\authorname} % Author name - remove the \href bracket to remove the link
\end{flushleft}
\end{minipage}
\begin{minipage}[t]{0.3\textwidth}
\begin{flushright} \large
\emph{Advisor:} \\
\href{https://profesores.virtual.uniandes.edu.co/mlinaresv/en/inicio-en/}{\supname} % Supervisor name - remove the \href bracket to remove the link  
\end{flushright}
\end{minipage}\\[1.5cm]
 
\vfill

\large \textit{A thesis submitted in fulfillment of the requirements\\ for the degree of \degreename}\\%[0.1cm] % University requirement text
\textit{in}\\%[0.3cm]
%\groupname
\begin{figure}[h]
\centering
\includegraphics[width=0.5\textwidth]{../Figures/tsdlLargo}
\end{figure}
\deptname\\[0.5cm] % Research group name and department name
 
\vfill

{\large \today}\\
%[4cm] % Date
%\includegraphics{F} % University/department logo - uncomment to place it
 
\vfill
\end{center}
\end{titlepage}


%----------------------------------------------------------------------------------------
%	ABSTRACT PAGE
%----------------------------------------------------------------------------------------

\begin{abstract}
\addchaptertocentry{\abstractname} % Add the abstract to the table of contents
The number of different tools to explore Android applications has been increasing. Every tool has a different exploration strategy and clam to offer different benefits than others. The huge amount of tools and the lack of impartial information about them makes that developers and researchers have no basis and data to face a decision-making situation or data to compare their own new tools. 
Others studies have made different comparisons between exploration tools in the past, but most of those tools are no longer being used in the industry or in the academy, that is why there is a need of studies providing clear and unbiased information about the newest tools that allows the developers and researchers to acquire a better perspective of the modern exploration tools. 
That is the reason why in this study, four of the most used tools for automatic exploration of Android applications are analysed and compared according their progressive and achieved method coverage, and the max number of errors found in one exploration. Besides, a reproducible workflow is proposed for future studies of the same type as well as two tools for allowing faster and easier comparison are described.
\end{abstract}

%----------------------------------------------------------------------------------------
%	ACKNOWLEDGEMENTS
%----------------------------------------------------------------------------------------

\begin{acknowledgements}
\addchaptertocentry{\acknowledgementname} % Add the acknowledgements to the table of contents

First, I want to express my deepest thanks to Professor Mario Linares-Vásquez for helping me with the development of this final work, giving me the necessary feedback for getting this project to this final version, and new ideas and ways to solve the presented problems while executing this research.

Second, I would like to give my thanks to all the members of The Software Design Lab, for sharing with me their experiences and knowledge that were very important for developing this thesis. Especially to Camilo Escobar for his great help giving the main concept of InstruAPK, for helping with its implementation, besides giving me feedback about the figures in this text as well as solving some extra questions and doubts that I had during the process of developing this thesis.

Third, I want to say thanks to my mother and sister for keeping me motivated within all my major, till the last moment of it. For their unconditional support and for being there in the moments I needed them the most. 

At last, but not least important, I want to say thanks to all my friends for sharing their knowledge with me, and contributing with that to this thesis. 

Without the help of the mentioned people, this work would not be possible.

I wish to clarify that the order in the mention does not reflect the level of thankfully I feel for the people mentioned in this statement. All of them supported this work in different ways, and under their capabilities,. Helping me in one way or another to reach this results. For that reason, all of them deserve the same feelings from my. One more time, thanks to all of them.

\end{acknowledgements}

%----------------------------------------------------------------------------------------
%	LIST OF CONTENTS/FIGURES/TABLES PAGES
%----------------------------------------------------------------------------------------

\tableofcontents % Prints the main table of contents

\listoffigures % Prints the list of figures

\listoftables % Prints the list of tables


%----------------------------------------------------------------------------------------
%	SYMBOLS

%----------------------------------------------------------------------------------------
%	THESIS CONTENT - CHAPTERS
%----------------------------------------------------------------------------------------

\mainmatter % Begin numeric (1,2,3...) page numbering

\pagestyle{thesis} % Return the page headers back to the "thesis" style

% Include the chapters of the thesis as separate files from the Chapters folder
% Uncomment the lines as you write the chapters

% Chapter Template

\chapter{Introduction} % Main chapter title

\label{Chapter1} % Change X to a consecutive number; for referencing this chapter elsewhere, use \ref{ChapterX}
 
Parafrasear lo que dice aquí, decir que este gran número de aplicaciones por fortuna a llevado a un gran número de estudios para mejorar la cobertura de código que se logra durante las pruebas automáticas, y también cubir diferentes estados del celular como modo avión con carga sin carga y demás, dada la gran cantidad de herramientas, los desarrolladores pueden sentirse sobrecargados, pueden exitir muchas opciones y pueden que no se elija la mejor herramienta. 
Los desarrolladores necesitan formas de elegir la mejor herramienta que se adapte lo mejor posible a sus necesidades.

-------------------------------------------------------------------------
 
Mobile markets have pushed and promoted the raising of an interesting phenomenon that has permeated not only developers culture, but also human beings’ daily life activities. Mobile devices, apps, and services are helping companies and organizations to make “digital transformation” possible through services and capabilities that are offered ubiquitously and closer to the users. Nowadays, mobile apps and devices are the most common way for accessing those services and capabilities; in addition, apps and devices are indispensable tools for allowing humans to have in their phones, computational capabilities that make life better and easier.

The mobile apps phenomenon has also changed drastically the way how practitioners design, code, and test apps.  Mobile developers and testers face critical challenges on their daily life activities such as (i) continuous pressure from the market for frequent releases of high quality apps, (ii) platform fragmentation at device and OS levels, (iii) rapid platform/library evolution and API instability, and (iv) an evolving market with millions of apps available for being downloaded by ends users \cite{joorabchi2013real,palomba2018crowdsourcing}. Tight release schedules, limited developer and hardware resources, and cross-platform delivery of apps, are common scenarios when developing mobile apps \cite{joorabchi2013real}. Therefore, reducing the time and effort devoted to software engineering tasks while producing high quality mobile software is a ``precious’’ goal.

Both practitioners and researchers, have contributed to achieve that goal, by proposing approaches, mechanisms, best practices, and tools that make the development process more agile. For instance, cross-platform languages and frameworks (e.g., Flutter, Ionic, Xamarin, React Native) contribute to reducing the development time by providing developers with a mechanism for building Android and iOS versions of apps in a write-one-run-anywhere way \cite{joorabchi2013real,fazzini2017automated}. Automated testing approaches help testers to increase the apps' quality and reduce the detection/reporting time \cite{choudhary2015automated,kochhar2015understanding,linares2017continuous}. 
Automated categorization of reviews also helps developers to select relevant information, issues, features and sentiments, from large volume of review that are posted by users \cite{palomba2018crowdsourcing,villarroel2016release,di2016would}. Moreover, approaches for static analysis, are helping developers to early detect different types of bugs and issues that without the automated support could be time consuming for developers --- when doing the analysis manually \cite{li:IST2017}. 
Both static and dynamic analyses have been used with the aforementioned approaches, with a special preference for static analysis on source code.  

The developers community is quickly moving towards using cloud-services and crowd-sourced services for software engineering tasks \cite{Leicht2017IEEESoftware, stol2017crowdsourcing}; using those services is becoming a common practice of mobile developers who want to reduce costs and the time devoted for an activity. For example, the Firebase Test Lab platform \cite{firebase} provides automated testing services, in particular, it automatically executes/explores a given app (provided by the developer as an Android APK file), and reports crashes found on a devices matrix that is selected by the user. However, the lack of knowledge of source code internals imposes a limitation on the usefulness and completeness of the results reported back to the users.

%----------------------------------------------------------------------------------------
%	SECTION 1
%----------------------------------------------------------------------------------------

\section{Problem Statement}

// TODO explicar el problema que se quiere solucionar. Para mi es el comparar las diferentes herramientas puede tomar tiempo y elegir la correcta para un proyecto o para una aplicación puede tomar tiempo valioso


\section{Thesis  Goals}

The main objective of this thesis, is to provide quantitative and qualitative information of the most widely used automatic exploration tools, to facilitate developers in the selection of the right tool that suits their needs. Under those circumstances, the next specific objectives were proposed.
		\begin{enumerate}
			\item Compare the tools by their exploration coverage 
			\item Compare the tools by the number of unique error traces discovered while exploring an application.
			\item Compare the tools using qualitative aspects such as, is the tool a open source project? Is the tool free? Is the tool allowing introduce login values? how useful is the tool report for developer to reproduce, find and fix bugs?
		\end{enumerate}




\section{Thesis contribution}

The main contribution of this thesis is to provide developers with enough information to decide which is the best automatic exploration tool for their projects.


%TODO agregar datos relacionados con las conclusiones diciendo lo que se concluyó. Por ejemplo que en la mayoría de los casos es mejor usar una herramienta que la otra y decir que a pesar de que el objetivo de la tesis no era realizar una herramienta para permitirles comparar, pues se creó una que le s va a permitir comparar comodamente las diferentes herramientas usando sus propios apks.

	
\section{Document Structure}

%TODO Hacer al final porque no se sabe la estructura antes de


% Chapter Template

\chapter{Related work} % Main chapter title


\label{Chapter3} % Change X to a consecutive number; for referencing this chapter elsewhere, use \ref{ChapterX}

%----------------------------------------------------------------------------------------
%	SECTION 1
%----------------------------------------------------------------------------------------

Every exploration tool provides different numbers, and comparisons made during their creation, this in order to show their advantages, but not for sure their weaknesses. This information does not allow developers neither researchers to know what are the best tools as of now or how good a tool matches their projects. 
%TODO Cite them propertly here.

Shauvik Roy Choudhary, et al., give about the strengths and weaknesses of seven tools in their study "Automated Test Input Generation for Android:
Are We There Yet?". They evaluate the tools using four metrics i. ease
of use, ii. ability to work on multiple platforms, iii. code coverage, and iv.
ability to detect faults. 14 Tools and 68 applications were used in total in their study. Running 10 times every application in seven of the 14 tools for a maximum of 60 minutes.

Moreover, different tools have been developed since Shauvik Roy Choudhary, et al. study was made. One example is Firebase Test Lab, which started to allowing the automatic testing in the cloud. This type of application was not taken into account.

% Chapter Template

\chapter{Solution Design} % Main chapter title

\label{Chapter4} % Change X to a consecutive number; for referencing this chapter elsewhere, use \ref{ChapterX}

\section{General Approach} \label{sec:generalApproach}

With the purpose of achieving the objectives mentioned in Sec.\ref{sec:thesisGoals}, a workflow was designed. This workflow contains five stages: (1) Instrumentation of the applications, (2) Exploration, (3) Coverage measurement, (4) Summarize data, and (5)  Data Analysis. The workflow is depicted in Figure \ref{fig:workflow}.

\begin{figure}[h]
\centering
\includegraphics[width=\textwidth]{../Figures/workflow.jpg}
\caption{Main Workflow}\label{fig:workflow}
\end{figure}

For the first stage, \textbf{InstruAPK} (Section \ref{sec:instruAPK}) was used to make the instrumentation of the applications. This tool only takes into account the methods under the package name of the application that is being analysed. As a result, methods from different libraries are not instrumented. The input for this stage is the original APK file, and the output is the instrumented APK together with the instrumentation report containing general information such as the file path, method's name, file name, and the method arguments, as well as a sequential number that will help us to know the total amount of instrumented methods and will work as their unique identifier.

The exploration was made with four different exploration tools, (1) Droidbot, (2) Monkey, (3) Firebase Test Lab, and (4) RIP. \MARIO{Add citations for each tool} Every tool was executed to explore  apps with a maximum time of 30 minutes. It is important to notice that, even when the max execution time seems to be short, it was enough for most of the analyzed applications. This was because of the size of the applications: if the application is small, then the coverage will increase rapidly, because a bug was found during the exploration, or even because the tool marks the exploration as done. 

This stage input is only the instrumented APK file, and its output is the exploration report that every tool provides. 
 
%TODO poner la referencia del comando adb shell para el logcat https://developer.android.com/studio/command-line/logcat

Droidbot, Monkey and Test Lab were selected because of their high use in the industry, and RIP was selected because it is an active project from The Software Design Lab. 

In stage 3, \textbf{CoverageAnalyzer (CA)} (Section \ref{sec:ca}) was used to make the coverage measurement and search for error lines. This stage inputs are the original APK, the instrumentation report and the instrumented APK, both from stage 1, and the logcat from stage 2. Its output is a report containing two method coverage measurements, the first one, calculated using the number of methods reported by APKAnalyzer, a tool provided in the Android SDK Tools \MARIO{Citate APKAnalyzer}, and the second one with the number of instrumented methods reported by InstruAPK.

%TODO poner referencia de APKAnalyzer https://developer.android.com/studio/command-line/apkanalyzer

Stages 2. and 3. were repeated 10 times for every application that was selected. The multiple executions are intending to get average values as well as comparable results along the different exploration tools. On top of that, the input for stage 4 are all the method coverage reports from of the stage 3 as well as the exploration report from stage 2. The output are the accumulated method coverage by each tool for each application, which was calculated taking the unique methods called over all the ten executions, the average accumulated method coverage over time, as well as the number of errors and its average found per application.

The final stage, i.e.,  Stage 5,  encompasses data understanding, graphs creation, comparison using the graph and analysis of different qualitative aspects of every exploration tool. Thus, \MARIO{....}


%TODO citate InstruAPK repository
%TODO citate CA repository

\section{InstruAPK}\label{sec:instruAPK}

This tool was developed for this study. It uses APKTool, a known Java application that allows inverse engineering of Android apps, allowing applications' instrumentation without the need of recompiling their source code. APKTool decodes the APK file and its result is the smali representation of the app source code. These smali files are analyzed in order to find all the methods to be instrumented at the very beginning of each method. It is important to notice that no external libraries methods are instrumented. InstruAPK only searches for methods following the android project structure that uses the application package name to store the application source code.

\begin{figure}[h]
\centering
\includegraphics[width=0.8\textwidth]{../Figures/ClassDiagramInstruAPK.jpg}
\caption{Class Diagram InstruAPK}\label{fig:instruAPK}
\end{figure}

Figure \ref{fig:instruAPK} only contains the main classes of the tool and offers a short explanation of what is doing every one to understand it in more detail.

\section{Coverage Analyser (CA)}\label{sec:ca}

This tool is a Java Application created for this study. It analyses the resulted logcat file after an exploration. It searches for the log lines injected by InstruAPK, as well as for errors, filtering the results using the package name of the application under the analysis. For the coverage measurement, the tool uses the number of methods reported by APKAnalyzer as well as the number of methods reported by InstruAPK, resulting in two different method coverage values. As mentioned before, CA searches for the log lines injected by InstruAPK making CA depend on it. For that reason, CA can be seen as a complement of InstruAPK, rather than a separate application.

%TODO Add apkanalyzer reference https://developer.android.com/studio/command-line/apkanalyzer

Figure \ref{fig:ca} contains the main classes of CoverageAnalyzer besides a short explanation of its functionality.

\begin{figure}[h]
\centering
\includegraphics[width=0.8\textwidth]{../Figures/ClassDiagramCA.jpg}
\caption{Class Diagram Coverage Analyser}\label{fig:ca}
\end{figure}

These two tools were the main basis of this study, but its further review its leave for previous studies.

Certainly, all the stages were necessary in order to complete every objective. The stages design was made regarding the specific objectives and as a result achieving the general one.

Any person who desires to compare different exploration tools, can reproduce this work flow. Even can make use of the same tools for the instrumentation and the coverage measurement, allowing easy and fast comparisons. Consequently, the decision-making starts to be easier for developers and researchers. Also,gives the possibility to researchers of compare their own exploration tools in a effortless and quick way.
% Chapter Template

\chapter{Empirical Study} % Main chapter title

\label{Chapter4} % Change X to a consecutive number; for referencing this chapter elsewhere, use \ref{ChapterX}

%----------------------------------------------------------------------------------------
%	SECTION 1
%----------------------------------------------------------------------------------------
\section{Study Design}\label{sec:studydesign}

The aim of this paper is to provide information about some of the most used automatic exploration tools for android applications inside the industry and the academy. This information is going to be useful for developers and researchers when they face a decision-making situation related to the selection of the right exploration tool that fits their needs. In consequence, a empirical study was designed and will provide answers for the following research questions: 

\begin{itemize}
\item RQ-1 What tool reaches the highest average method coverage?
\item RQ-2 What tool reaches the highest accumulated method coverage?
\item RQ-3 What tool finds the highest number of failures during the explorations?
\item RQ-4 what tool has the highest average of found errors?
\end{itemize}

\section{Context of the Study}

With the purpose of answering the research questions, 11 applications where selected to be executed. The list of selected apps can be seen in Figure.\ref{tab:apps}. This set is a subset of a set of open source applications utilised inside The Software Design Lab research group for other studies and tests, including RIP. Every APK in the subset should be successfully instrumented by InstruAPK, it should compile without any problem after instrumentation and it should be launch in an emulator without any issue after the instrumentation process.

Equally important, four exploration tools where selected, two from the industry and two from the academic side. The first tool was Firebase Test Lab (Section \ref{sec:testlab}). it was selected for being widely used in industry and for also being a Google product. The second one, Monkey \footnote{https://developer.android.com/studio/test/monkey}, was selected for being the most basic one. It is by default included in the Android SDK Tools. The third one, Droidbot (Section \ref{sec:droidbot}), was selected from the academic side. Droidbot has been a point of study for many researches. Many others tools have based their functionality on this tool. The last one is RIP (Section \ref{sec:rip}), this tool was selected for being of special interest for us. It is our own exploration tool and is is currently an active project inside the Software Design Lab at University of Los Andes. 

Every tool was executed ten times per application, and every execution with a maximum time of 30 minutes. Some tools ended its exploration before the max time. 
The number of executions and the maximum time were arbitrary decisions that were made because of time limitations for the study. Although, during the study was notice that most of the tools ended the exploration or reached their maximum coverage within the first 15 minutes. Which means that the maximum time for exploration was more than enough in almost all cases. 

The same emulator was used for all the exploration tools, in exception to Firebase Test Lab because this tool offers its own set of emulators. In all the cases a Pixel 2 XL was used, but in the cases of Monkey, Droidbot and RIP there was more control over the device specifications.For the case of the last mentioned tools the specifications are: Google APIs Intel Atom (x86), API level 27, SD card size of 512MB, and RAM size of 4096MB. This specifications are unknown for the case of Test Lab.

\begin{table}[t]
	\centering
	\caption{Applications used for the study}
	\label{tab:apps}
	\resizebox{0.80\textwidth}{!}{
		\begin{tabular}{c c c c}
			App ID & Package Name & \# Methods Reported by APKAnalyzer & \# Methods Instrumented by InstruAPK\\
			\hline
			1 & appinventor.ai\_nels0n0s0ri0.MiRutina & 61993 & 9351\\
			2 & com.evancharlton.mileage & 4000 & 1162\\
			3 & com.fsck.k9 & 18799 & 7003\\
			4 & com.ichi2.anki & 32370 & 2209\\
			5 & com.workingagenda.devinettes & 19274 & 66 \\
			6 & de.vanitasvitae.enigmandroid & 13083 & 574 \\
			7 & info.guardianproject.ripple & 19429 & 100 \\
			8 & org.connectbot & 20606 & 1145\\
			9 & org.gnucash.android & 75473 & 504\\
			10 & org.libreoffice.impressremote & 14691 & 649\\
			11 & org.lumicall.android & 45784 & 540\\
			\hline
		\end{tabular}
	}
\end{table}

Finally, the work flow specified in Section \ref{sec:generalApproach} was follow for every application using the named exploration tools. 

\section{Results}\label{sec:results}

\subsection{Method Coverage Results}\label{sec:coverageResults}

As mentioned before, some explorations ended before the max execution time allowed (30 minutes). giving no data for the upcoming seconds. To solve this scenarios, the coverage reached by the tool in the second the exploration ends, was kept the same until complete the total time. Once this have been done, the results are comparable second by second.


As a result of the instrumentation made by InstruAPK, the timestamp of every called method is know, so that, the coverage reached by a tool in a specific second can be calculated. The aforementioned information was used to calculate the average method coverage reached for every tool. Such results are presented in figure \ref{fig:averageCoverageInstruAPK} and figure \ref{fig:averageCoverageAPKAnalyzer}. The data were calculated as follow: 

\begin{enumerate}
\item For every exploration of an application add the coverage reached for tool second by second without adding the results of the previous seconds.
\item Divide by the number of explorations by tool. (This results in the average method coverage reached by a tool in one application)
\item Add the results of the previous step. (By tool)
\item Divide by the number of applications (This will end in the average method coverage of a tool for all the applications)
\end{enumerate} 

For calculate this, both data was used, the method coverage according InstruAPK and APKAnalyzer. The curves, have the same behaviour, as expected, but in Figure \ref{fig:averageCoverageAPKAnalyzer} the gap between tools is more visible.

\begin{figure}[h]
\centering
\includegraphics[width=0.8\textwidth]{../Figures/averageCoverageInstruAPK.pdf}
\caption{Average Method Coverage by Tool According InstruAPK}\label{fig:averageCoverageInstruAPK}
\end{figure}

\begin{figure}[h]
\centering
\includegraphics[width=0.8\textwidth]{../Figures/averageCoverageAPKAnalyzer.pdf}
\caption{Average Method Coverage by Tool According APKAnalyzer}\label{fig:averageCoverageAPKAnalyzer}
\end{figure}

The answer for RQ-1 can be seen easily in figure \ref{fig:averageCoverageInstruAPK} or figure \ref{fig:averageCoverageAPKAnalyzer}. Firebase Test Lab is the tool with the highest average method coverage reached. Surprisingly, followed by Monkey, even when monkey has no complex architecture nor exploration strategy, it has the second highest average method coverage within this study.

For answering RQ-2 is know that, for every execution the called methods are stored, every method has an unique id that was given during the instrumentation. Thus, after the 10 executions of an application, the result of every execution is analysed searching for the methods that have not been called neither during previous explorations nor during the current one. Therefore, the number of unique methods called in that application is obtained. That is what is named as accumulated coverage. This results can be seen in figure \ref{fig:boxplotAccumulated}. The coverage presented in figure \ref{fig:boxplotAccumulated} was calculated using the number of instrumented methods reported by InstruAPK. 

\begin{figure}[h]
\centering
\includegraphics[width=0.8\textwidth]{../Figures/boxplotAccumulated.pdf}
\caption{Boxplot of Accumulated Coverage by Tool}\label{fig:boxplotAccumulated}
\end{figure}

So, again, the tool at first place is Firebase Test Lab, but this time, the second tool with the highest value is Droidbot.

\subsection{Error Results}\label{sec:errorResults}

Two different values for the errors were calculated. first, the total number of different error traces found by a tool throughout all the explorations. That is what is shown in figure \ref{fig:maxerrors}. The second one, show in figure \ref{fig:averagaerrors}, is the average number of errors found by tool throughout all the explorations. 

\begin{figure}[h]
\centering
\includegraphics[width=0.8\textwidth]{../Figures/maxErrors.pdf}
\caption{Maximum Number of Errors Found by Tool}\label{fig:maxerrors}
\end{figure}

\begin{figure}[h]
\centering
\includegraphics[width=0.8\textwidth]{../Figures/averageErrors.pdf}
\caption{Average Number of Errors Found by Tool}\label{fig:averagaerrors}
\end{figure}

This numbers were extracted by CoverageAnalyzer (Section \ref{sec:ca}). The number includes android runtime exceptions as well as exceptions. The tool searches for stack traces in the logcat file and filters the results using the package name of the tool under analysis. 

This data give the responses for RQ-3 and RQ-4. Which in both cases is Monkey.


% Chapter Template

\chapter{Conclusion} % Main chapter title

\label{ChapterConclusion} % Change X to a consecutive number; for referencing this chapter elsewhere, use \ref{ChapterX}


%----------------------------------------------------------------------------------------
%	BIBLIOGRAPHY
%----------------------------------------------------------------------------------------

\printbibliography

%----------------------------------------------------------------------------------------

\end{document}  
